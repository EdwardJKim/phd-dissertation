%%
%% This is file `thesis-ex.tex',
%% generated with the docstrip utility.
%%
%% The original source files were:
%%
%% uiucthesis2014.dtx  (with options: `example')
%% 
\def\fileversion{v2.25b} \def\filedate{2017/05/02}
%% Package and Class "uiucthesis2014" for use with LaTeX2e.
\documentclass[12pt,edeposit,fullpage]{uiucthesis2014}

% references at the end of each chapter
\usepackage[round,sectionbib]{natbib}
\usepackage{chapterbib}

\usepackage{amsmath}
\usepackage{amssymb}
\usepackage{bibentry}
\usepackage{graphicx}
\usepackage{hyperref}
\usepackage{caption}
\usepackage{subcaption}
\usepackage{rotating}

\def\aap{{A\&A}}
\def\AaA{{A\&A}}
\def\AJ{{AJ}}
\def\aj{{AJ}}
\def\ApJ{{ApJ}}
\def\apj{{ApJ}}
\def\apjl{{ApJL}}
\def\ApJL{{ApJL}}
\def\ApJSupp{{ApJS}}
\def\MN{{MNRAS}}
\def\mnras{{MNRAS}}
\def\mnrasl{{MNRAS Letters}}
\def\pasp{{PASP}}
\def\jcap{JCAP}
\def\pasj{{PASJ}}
\def\nat{Nature}
\def\ARAA{{ARA\&A}}
\def\apjs{{ApJS}}
\def\araa{{ARA\&A}}
\def\apss{{Ap\&SS}}
\def\cjaa{{ChJAA}}
\def\aaps{{A\&AS}}

\newcommand{\eg}{{e.g., }}
\newcommand{\ie}{{i.e., }}
\newcommand\blfootnote[1]{%
  \begingroup
  \renewcommand\thefootnote{}\footnote{#1}%
  \addtocounter{footnote}{-1}%
  \endgroup
}

% put figure on a new page
\renewcommand{\floatpagefraction}{0.1}

\nocopyrightpage

\begin{document}

\title{Machine Learning Approaches to\\
       Star-galaxy Classification}
\author{Junhyung Kim}
\department{Physics}
\phdthesis
\advisor{Robert J.\ Brunner}
\degreeyear{2018}
\committee{
    Professor Emeritus Jon J.\ Thaler, Chair\\
    Professor Robert J.\ Brunner, Director of Research\\
    Professor George D.\ Gollin\\
    Assistant Professor Alexander G.\ Schwing}
\maketitle

\frontmatter

%% Create an abstract that can also be used for the ProQuest abstract.
%% Note that ProQuest truncates their abstracts at 350 words.
\begin{abstract}
Accurate star-galaxy classification has many important applications in modern precision cosmology.
However, a vast number of faint sources that are detected in the current and
next-generation ground-based surveys may be challenged by poor star-galaxy classification.
Thus, we explore a variety of machine learning approaches to improve
star-galaxy classification in ground-based photometric surveys.
In Chapter~\ref{chapter2}, we present a meta-classification framework that combines existing star-galaxy classifiers,
and demonstrate that our Bayesian combination technique improves the overall performance over any individual classification method.
In Chapter~\ref{chapter3}, we show that a deep learning algorithm called convolutional neural networks
is able to produce accurate and well-calibrated classifications by learning directly from the pixel values of photometric images.
In Chapter~\ref{chapter4}, we study another deep learning technique called generative adversarial networks
in a semi-supervised setting, and demonstrate that our semi-supervised method produces competitive classifications
using only a small amount of labeled examples.

\end{abstract}

%% Create a dedication in italics with no heading, centered vertically
%% on the page.
%\begin{dedication}
%To Father and Mother.
%\end{dedication}

%% Create an Acknowledgements page, many departments require you to
%% include funding support in this.
\chapter*{Acknowledgments}

First and formorest, I would like to thank my advisor, Robert Brunner, for his guidance and support
during my graduate studies at UIUC.
I also wish to thank my committee members, Jon Thaler, George Gollin, and Alex Schwing, for their help and support.
I am also grateful to Matias Carrasco Kind, Yiran Wang, Nacho Sevilla, William Biscarri, Samantha Thrush, Xinyang Lu, and Kelechi Ikegwu.
A special thanks to my father and my wife who endured this long process with me.

\begin{flushright}
--- Edward Junhyung Kim
\end{flushright}

\vspace{10mm}

This work was supported in part by the National Science Foundation Grant
No.\ AST-1313415.
This work used the Extreme Science and Engineering Discovery Environment
(XSEDE), which is supported by National Science Foundation grant number
ACI-1053575.
This work is based on observations obtained with MegaPrime/MegaCam, a
joint project of CFHT and CEA/DAPNIA, at the Canada-France-Hawaii
Telescope (CFHT) which is operated by the National Research Council
(NRC) of Canada, the Institut National des Sciences de l'Univers of
the Centre National de la Recherche Scientifique (CNRS) of France, and
the University of Hawaii. This research used the facilities of the
Canadian Astronomy Data Centre operated by the National Research
Council of Canada with the support of the Canadian Space Agency.
CFHTLenS data processing was made possible thanks to significant
computing support from the NSERC Research Tools and Instruments grant
program.
Funding for the DEEP2 survey has been provided by NSF grants AST-0071048,
AST-0071198, AST-0507428, and AST-0507483. 
Funding for SDSS-III has been provided by the Alfred P. Sloan Foundation, the
Participating Institutions, the National Science Foundation, and the U.S.
Department of Energy Office of Science. The SDSS-III web site is
http://www.sdss3.org/.
This work used data from the 
VIMOS Public Extragalactic Redshift Survey (VIPERS).
VIPERS has been performed using the ESO Very Large Telescope, under the "Large
Programme" 182.A-0886. The participating institutions and funding agencies are
listed at http://vipers.inaf.it/.
This work used data from the VIMOS VLT Deep Survey, obtained from the VVDS
database operated by Cesam, Laboratoire d'Astrophysique de Marseille, France.

%% The thesis format requires the Table of Contents to come
%% before any other major sections, all of these sections after
%% the Table of Contents must be listed therein (i.e., use \chapter,
%% not \chapter*).  Common sections to have between the Table of
%% Contents and the main text are:
%%
%% List of Tables
%% List of Figures
%% List Symbols and/or Abbreviations
%% etc.

\tableofcontents
%\listoftables
%\listoffigures

%% Create a List of Abbreviations. The left column
%% is 1 inch wide and left-justified
%\chapter{List of Abbreviations}
%
%\begin{symbollist*}
%\item[CA] Caffeine Addict.
%\item[CD] Coffee Drinker.
%\end{symbollist*}

%% Create a List of Symbols. The left column
%% is 0.7 inch wide and centered
%\chapter{List of Symbols}
%
%\begin{symbollist}[0.7in]
%\item[$\tau$] Time taken to drink one cup of coffee.
%\item[$\mu$g] Micrograms (of caffeine, generally).
%\end{symbollist}

\mainmatter

\chapter{Introduction}


\section{Star-galaxy Classification in Photometric Surveys}


Currently ongoing and upcoming large-scale surveys, such as the Dark Energy Survey (DES)
and the Large Synoptic Survey Telescope (LSST), are purely photometric surveys,
where digital images of the sky are obtained and subsequently analyzed.
To quantify the brightness of a source in a photometric image,
we count the number of photons from the source within a fixed aperture
(\eg a circle or a two-dimensional Gaussian).
This brightness measurement is expressed in units of magnitude ---
a logarithmic unit in which the fainter a source appears the larger its magnitude.
In mathematical terms, the apparent magnitude $m$ in a spectral band $\lambda$ is given by
\begin{equation}
m_{\lambda} = - 2.5 \log_{10} \frac{ F_{\lambda} }{ F_{\lambda,0} },
\end{equation}
where $F_{\lambda}$ is the observed flux using the photometric filter $\lambda$ ,
and $F_{\lambda,0}$ is the reference flux (\ie zero-point) for that filter.
Photometric surveys use filters on telescopes to allow only light around
a specific wavelength to pass.
Figure~\ref{fig:filters} shows the wavelengths of the five filters (named $u$, $g$, $r$, $i$, $z$)
of the Sloan Digital Sky Survey (SDSS).
Before these photometric data can be used for a scientific analysis, however, they must be classified,
which for most sources is either a star or a galaxy.

Stars are in our Milky Way galaxy and are close to us compared to distant galaxies.
Due to their small physical size, however, almost all stars appear as compact point sources
in photometric images.
Galaxies, despite being farther away, generally subtend a larger angle, and thus
appear as extended sources.
However, as Figure~\ref{fig:sg_mag} demonstrates, it becomes increasingly difficult
to separate stars from galaxies due to a large number of unresolved galaxies at faint magnitudes.
Since the number of galaxies grows exponentially with magnitude, 
this implies that the majority of sources that are detected in the current and
next-generation ground-based surveys may be challenged by poor star-galaxy classification.
Furthermore, due to the sheer number of stars and galaxies,
this classification has to be automated.
For example, the SDSS has obtained photometric observations of more than $3 \times 10^8$ objects
\citep{eisenstein2011sdss},
and the LSST will produce a catalog of $2 \times 10^{10}$ galaxies and a similar number of stars
\citep{ivezic2008lsst}.
Thus, there is a need for a robust, automated classification technique for large ground-based photometric surveys.

The classification of stars vs.\ galaxies has many important applications in precision cosmology.
As a basic example, in a homogeneous universe with a Euclidean geometry for three-dimensional space,
the number counts of galaxies as a function of magnitude follows
\begin{equation}
N \left( m_{\lambda} \right) \propto 10^{ 0.6 \left( m_{\lambda} - m_{\lambda,0} \right) }.
\end{equation}
By comparing this relation with the predictions of a Friedmann-Robertson-Walker (FRW) universe
(\ie the standard model of cosmology),
\citet{yasuda2001galaxy} show that our universe does not have a Euclidean geometry for
three-dimensional space.
Without a reliable method for separating stars from unresolved galaxies, 
we risk underestimating the number density of galaxies by rejecting all unresolved galaxies,
while including them could result in significant contamination of the galaxy sample.
Furthermore, the accurate separation of stars and galaxies in faint samples
significantly improves our ability to
(i) measure auto-correlation functions of luminous galaxies \citep{ross2011ameliorating},
(ii) control the systematic errors in the weak lensing shear measurement \citep{soumagnac2015star},
(iii) map the signature of baryon acoustic oscillations \citep{anderson2014clustering}, and
(iv) identify electromagnetic counterparts to gravitational wave sources \citep{miller2017preparing},
among other things.

Given the importance of this classification problem, it is not surprising that a variety of
different strategies have been developed.
The most commonly used method to classify stars and galaxies in large sky surveys is the morphological separation
\citep{sebok1979optimal, kron1980photometry, valdes1982resolution, yee1991faint, vasconcellos2011decision,
henrion2011bayesian}.
It relies on the assumption that stars appear as point sources while galaxies appear as resolved sources.
For example, a popular technique in the weak lensing community \citep{Kaiser1995}
makes a hard cut in the space of photometric attributes as shown in Figure \ref{fig:intro_morc}.
As the Figure shows, there is a distinct locus
produced by point sources in the half-light radius vs.\ the $i$-band magnitude plane.
(The half-light radius is the effective radius at which half of the total light of an object is contained.)
A rectangular cut in this size-magnitude plane separates point sources
(which are presumed to be stars) from resolved sources (which are presumed to be galaxies).

However, such a hard cut in a low-dimensional parameter space has disadvantages:
it does not break down gracefully; its treatment of measurement uncertainties is too simplistic;
it uses a rather limited subset of the full information available;
and it ignores a priori information like the expected demographics of the source populations.
Furthermore, currently ongoing and upcoming large photometric surveys
will detect a vast number of unresolved galaxies at faint magnitudes.
Near a survey's limit, the photometric observations cannot reliably separate stars from unresolved galaxies
by morphology alone without leading to incompleteness and contamination in the star and galaxy samples.

\section{Machine Learning}

\subsection{Supervised Learning}

The systematic misclassification of sources can be mitigated by using machine learning algorithms.
Machine learning methods have the advantage that it is easier to include extra information,
such as shape information or different model magnitudes.
Machine learning techniques are usually categorized into two main types: supervised and unsupervised learning approaches.
In the supervised learning approach, the input attributes (\ie the values that describe the properties of each objects \eg magnitudes),
$\mathbf{X}= \{ \mathbf{x}_1,\mathbf{x}_2,\dots,\mathbf{x}_N \}$,
are provided along with the desired output values (\eg star or galaxy),
$\mathbf{y} = \{ y_1, y_2, \dots, y_N \}$, in a labeled set of input-output pairs
$\mathbf{D} = \{ \left( \mathbf{x}_i, y_i \right) \}_{i = 1}^N$.
Here, $\mathbf{D}$ is the training set, and $N$ is the number of training examples.
The goal of supervised learning is then to estimate a function that maps $f: \mathbf{X} \rightarrow \mathbf{y}$.
As we discuss in the following chapters, it is desirable for the algorithm to return a probability.
To emphasize the need for probabilistic predictions, we formulate the goal of supervised learning as folows:
a probabilistic supervised learning algorithm infers the probability distribution $P( \mathbf{y} | \mathbf{X}, \mathbf{D} )$
over possible labels, given the input $\mathbf{X}$ and training set $\mathbf{D}$.
We use the conditioning bar $|$ to explicitly show that the probability is conditional on both
the input $\mathbf{X}$ and the training set $\mathbf{D}$.
When we have a set of multiple models to choose from, we explicitly condition the probability on the set of models and write
$P \left( \mathbf{y} | \mathbf{X}, \mathbf{D}, \mathbf{M} \right)$, where $\mathbf{M}$ is the set of models.
However, if it is clear from the context which model we use to make predictions, we drop $\mathbf{M}$ although
it is implied that the probability is conditional on the form of model.

To obtain the truth labels for the training data,
we use spectroscopy to measure the spectrum of electromagnetic radiation
from stars and galaxies.
Although modern spectrometers are more complex,
a spectrometer, in its most basic form, consists of a slit,
a prism or diffraction grating (to split the light into its component colors), and a detector.
We can use spectroscopy to measure many properties of distant stars and galaxies,
such as their chemical composition, temperature, and distance, and thus
spectral classification can be used as the ground truth for classifying sources in photometric images.

\subsection{Neural Networks}

As an example of a supervised learning algorithm, we provide a brief description of
\textit{artificial neural networks} (ANN)---the most widely used machine learning algorithm in astronomy.
The use of neural networks in astronomy goes as far back as the mid 1980s \citep{jeffrey1986optimization}.
ANN was first applied to the star-galaxy classification problem by \citet{odewahn1992automated},
and it has become a core part of the popular astronomical image processing software \textsc{SExtractor}~\citep{bertin1996sextractor}.

The original motivation for ANNs was to simulate neurons in the human brain.
A neuron in the human brain receives signals from other neurons through synaptic connections.
If the combination of these signals exceeds a certain threshold,
the neuron will fire and send a signal to other neurons.
Intelligence is believed to be the collective effect of
approximately $10^{11}$ neurons firing.
An artificial neuron in most artificial neural networks is represented
as a mathematical function that models a biological neural structure
(Figure~\ref{fig:intro_neuron_a}).
Let $\mathbf{x}=\left(x_1,x_2,\dots,x_n\right)$ be a vector of inputs to a given neuron,
$\mathbf{w}=\left(w_1,w_2,\dots,w_n\right)$ be a vector of weights, and
$b$ be the bias.
Then, the output of the neuron is
\begin{equation}
  y = \sigma \left( \mathbf{w} \cdot \mathbf{x} + b \right),
  \label{eq:intro_neuron_output}
\end{equation}
where $\sigma$ is the activation function (or \textit{non-linearity}).
Common activation functions include the sigmoid function,
\begin{equation}
\sigma(x)=1/\left(1+e^{-x}\right),
\end{equation}
the hyperbolic tangent function,
\begin{equation}
\sigma(x)=\tanh(x),
\end{equation}
and the rectified linear unit \citep[ReLU;][]{nair2010rectified},
\begin{equation}
\sigma(x)=\max(0, x).
\end{equation}
Typical neurons are organized as layers, where each neuron in one layer is connected to the neurons of the subsequent layer.
A schematic representation is shown in Figure~\ref{fig:intro_neuron_b}.
All layers except the input and output layers are conveniently called hidden layers.

The training uses an algorithm to a set of weights and biases such that, given $N$ samples, the output from the network
$\mathbf{y}=\left(\hat{y}_1, \hat{y}_2, \dots, \hat{y}_N \right)$
approximates the desired output
$\mathbf{y} = \left(y_1, y_2, \dots, y_N \right)$
as closely as possible for all input
$\mathbf{X}=\left(\mathbf{x}_1,\mathbf{x}_2,\dots,\mathbf{x}_N\right)$.
We can formulate this as the minimization of a loss function $L(\mathbf{y},\hat{\mathbf{y}})$ over the training data.
A common form of the loss function is the \textit{cross-entropy},
\begin{equation}
  L(y_j, \hat{y}_j) = - \frac{1}{N} \sum_{j=1}^{N} y_j  \log_2 \hat{y}_j
    + (1 - y_j)  \log_2 (1 - \hat{y}_j).
  \label{eq:intro_cross_entropy}
\end{equation}
where $y_j$ is the actual truth value (\eg 0 or 1) of the $j$-th data, and
$\hat{y}_j$ is the probability prediction made by the model.

To find the weights $\mathbf{w}$ and biases $\mathbf{b}$ which minimize the loss,
we use a technique called \textit{gradient descent},
where we use the following rules to update the parameters in each layer $l$:
\begin{align}
  \mathbf{w}_l &\rightarrow
  \mathbf{w}_l^{\prime}
  = \mathbf{w}_l - \eta \frac{\partial L}{\partial \mathbf{w}_l} \nonumber \\
  \mathbf{b}_l &\rightarrow
  \mathbf{b}_l^{\prime}
  = \mathbf{b}_l - \eta \frac{\partial L}{\partial \mathbf{b}_l},
  \label{eq:intro_gradient_descent}
\end{align}
where $\eta$ is a small, positive number known as the \textit{learning rate}.
The gradients in \ref{eq:intro_gradient_descent} can be computed using the
\textit{backpropagation} procedure~\citep{rumelhart1988learning},
which is nothing more than an application of the chain rule for derivatives.

\subsection{Unsupervised and Semi-supervised Learning}

In contrast to supervised learning, in which the truth labels are provided,
unsupervised learning does not utilize the desired output during the learning process.
Instead, we are only given unlabeled inputs $\mathbf{D}= \{ \mathbf{x}_i \}_{i=1}^N$,
and the data is clustered into different classes or categories.
In other words, unsupervised learning attempts to infer the probability distribution of the form $P(\mathbf{x}_i)$.
Unsupervised machine learning techniques are less common, in part due to the successes of purely supervised learning.
Semi-supervised learning falls between supervised learning, where training data are completely labeled,
and unsupervised learning, where all training data are unlabeled.
Semi-supervised techniques make use of a large amount of unlabeled data, in conjunction with a small amount of labeled data,
to better capture the underlying data distribution.
We expect unsupervised and semi-supervised learning to become more important, since
it is unclear if sufficient training data will be available in future ground-based photometric surveys and
there will be many orders of magnitude more unlabeled than labeled data available in future ground-based imaging surveys.

\section{Thesis Structure}

In the following chapters, we explore a variety of statistical and machine learning approaches to push the limits of
star-galaxy classification in ground-based photometric surveys.
Each chapter is self-contained and has its own references.

In Chapter~\ref{chapter2}, we present a novel meta-classification
framework that combines and fully exploits different techniques
to produce a more robust star-galaxy classification.
To demonstrate this hybrid, ensemble approach,
we combine a purely morphological classifier,
a supervised machine learning method based on random forest,
an unsupervised machine learning method based on self-organizing maps,
and a hierarchical Bayesian template fitting method.
Using data from the Canada-France-Hawaii Telescope Lensing Survey (CFHTLenS),
we consider different scenarios:
when a high-quality training set is available with spectroscopic labels,
and when the demographics of sources in a low-quality training set
do not match the demographics of objects in the test data set.
We demonstrate that our Bayesian combination technique improves
the overall performance over any individual classification method
in these scenarios.

In Chapter~\ref{chapter3}, we present a star-galaxy classification framework that uses a supervised machine learning algorithm called
convolutional neural networks (ConvNets).
Most existing star-galaxy classifiers use the reduced summary information from catalogs,
requiring careful feature extraction and selection.
Deep ConvNets allow a machine to automatically learn the features directly from images,
minimizing the need for input from human experts.
Using data from the SDSS and CFHTLenS,
we demonstrate that ConvNets are able to produce accurate and well-calibrated probabilistic classifications that are competitive with
conventional machine learning techniques.

In Chapter~\ref{chapter4}, we study the application of a deep learning technique called generative adversarial networks (GANs)
to the star-galaxy classification problem in a semi-supervised setting.
As current and forthcoming photometric surveys probe large cosmological volumes,
the majority of photometric observations are too faint for a uniform spectroscopic follow-up.
As a result, the number of unlabeled data available for training machine learning algorithms will be orders of magnitude
greater than the number of labeled data.
Semi-supervised learning techniques are of great interest since they are able to capture the underlying data distribution
with only a small amount of labeled data.
Using photometric images from the SDSS, we demonstrate that semi-supervised GANs are able to produce
accurate and well-calibrated classifications using only a small amount of labeled examples.
We also show that the number count distributions of the images generated by GAN follow a similar distribution to
the SDSS photometric sample.

In Chapter \ref{chapter5}, we outline our conclusions.

\newpage
\section{Figures and Tables}

\vspace{100pt}

\begin{figure}[htp]
  \centering
  \includegraphics[width=0.8\textwidth]{figures/filters.pdf}
  \caption{The SDSS $ugriz$ filter transmission curves.}
  \label{fig:filters}
\end{figure}

\begin{sidewaysfigure}[htp]
  \centering
  \includegraphics[width=\textwidth]{figures/sgstamps.pdf}
  \caption{Sample images of stars (top row) and galaxies (bottom row) from the SDSS survey
  at different magnitudes ($r$-band).
  Note that it becomes increasingly difficult to classify sources at fainter magnitudes,
  where we have the majority of the detected sources.}
  \label{fig:sg_mag}
\end{sidewaysfigure}

\begin{figure}[htp]
  \centering
  \includegraphics{figures/morph.pdf}
  \caption{Half-light radius vs.\ magnitude.}
  \label{fig:intro_morc}
\end{figure}

\begin{sidewaysfigure}
  \centering
  \begin{subfigure}[]{0.49\linewidth}
    \centering
    \includegraphics[width=0.6\textwidth]{figures/neuron.pdf}
    \caption{}
    \label{fig:intro_neuron_a}
  \end{subfigure}
  \begin{subfigure}[]{0.49\linewidth}
    \centering
    \includegraphics[width=0.6\textwidth]{figures/network.pdf}
    \caption{}
    \label{fig:intro_neuron_b}
  \end{subfigure}
  \caption{
    (a) A mathematical model of a biological neuron.
    (b) A schematic diagram of a neural network with one hidden layer.
    }
\end{sidewaysfigure}


\clearpage
\bibliographystyle{plainnat}
\bibliography{thesisbib}
\include{2-kim2015hybrid}
\include{3-kim2017star}
\include{4-kim2017ssgan}
\chapter{Conclusions}
  \label{chapter5}
 
\section{Summary and Conclusions}

In Chapter \ref{chapter2}, we have presented and analyzed a novel star-galaxy classification framework
for combining star-galaxy classifiers using the CFHTLenS data.
In particular, we use four independent classification techniques:
a morphological separation method;
TPC, a supervised machine learning technique
based on prediction trees and a random forest;
SOMc, an unsupervised machine learning approach
based on self-organizing maps and a random atlas;
and HB, a Hierarchical Bayesian template-fitting method
that we have modified and parallelized.
Using data from the CFHTLenS survey,
we have considered different scenarios:
when an excellent training set is available with spectroscopic labels from
DEEP2, SDSS, VIPERS, and VVDS, and
when the demographics of sources in a low-quality training set
do not match the demographics of objects in the test data set.
We demonstrate that the Bayesian Model Combination (BMC) technique improves
the overall performance over any individual classification method
in both cases.

The problem of star-galaxy classification is a rich area for future research. It is unclear
if sufficient training data will be available in future ground-based surveys. Furthermore, in
large sky surveys such as DES and LSST, photometric quality is not uniform across the
sky, and a purely morphological classifier alone will not be sufficient, especially at faint
magnitudes. Given the efficacy of our approach, classifier combination strategies are likely
the optimal approach for currently ongoing and forthcoming photometric surveys. Future
studies could apply the combination technique described in Chapter \ref{chapter2} to other surveys
such as the DES. Our approach can also be extended more broadly to classify objects that
are neither stars nor galaxies (e.g., quasars). Finally, future studies could explore the use
of multi-epoch data, which would be particularly useful for the next generation of synoptic
surveys.

In Chapter \ref{chapter3}, we have presented a convolutional neural network for classifying stars and
galaxies in the SDSS and CFHTLenS photometric images.
For the CFHTLenS data set, the network is able to provide a classification that
is as accurate as a random forest algorithm (TPC), while the probability estimates of
our ConvNet model appear to be better calibrated.
When the same network architecture is applied to the SDSS data set,
the network fails to outperform TPC,
but the probabilities are still slightly better calibrated.
The major advantage of ConvNets is that useful features are learned
automatically from images, while traditional machine learning
algorithms require feature engineering as a separate process
to produce accurate classifications.

Deep learning is a rapidly developing field, and recent developments include
improved network architectures.
Future work could explore more ConvNet variants, such as the
Inception Module~\citep{szegedy2015going} and Residual Network~\citep{he2015deep}.
To improve the predictive performance,
we have combined the predictions of different models across multiple
transformations of the input images (Section~\ref{sec:bmc}).
To further improve the performance, we could also train several networks
with different architectures and combine the models.
For example, the winning solution of \cite{dieleman2015rotation}
for the Galaxy Zoo challenge was based on a ConvNet model,
and it required averaging many sets of predictions from models with different
neural network architectures.
It is also likely that the performance will be improved
not only by training multiple network architectures,
but also by combining them with different star-galaxy classifiers.
ConvNets could be included as a different machine learning paradigm in the
classifier combination framework to produce further improvements in
predictive performance.

Our ConvNet model is a supervised algorithm, and one of the criticisms
of supervised techniques is their difficulty in extrapolating past the limits
of available spectroscopic training data.
Since it is difficult to assess the classification performance without a deeper
spectroscopic sample, we evaluated the performance using a test set
that is drawn from the same underlying distribution as the spectroscopic sample.
However, when our ConvNet model---trained on sources from a spectroscopic sample--- is
applied to a photometric sample---which is often fainter than our training set---the
performance of ConvNet will be less reliable.
Combining our ConvNet model with unsupervised methods (\eg a template fitting
method) in the aforementioned meta-classification framework will
help improve the efficacy of star-galaxy classification
beyond the limits of spectroscopic training data.

In Chapter \ref{chapter4}, we have presented a semi-supervised generative adversarial network for classifying stars, galaxies, and quasars in the SDSS photometric images.
We have demonstrated that the brightness and size distributions of images generated by our generative model
are in good agreement with those of the SDSS photometric images.
However, unlike most work on GANs, our focus was not solely on the generation of realistic images.
By using a small number of labeled images in conjunction with a large amount of unlabeled training data,
we have shown that our semi-supervised GAN is able to provide a classification that is comparable to the state-of-the-art
supervised methods.
we have also demonstrated the use of various scientific tools to validate our deep generative model.
In astronomy, we have powerful techniques for characterizing classifications, even in the absence of spectroscopic labels.
In contrast, most of the data sets used in the deep learning community are composed of natural images and text corpuses,
which lack such statistical techniques,
and direct comparison between different generative models is often difficult 
\citep{theis2016note}.
As a result, Astronomy has the potential to provide robust frameworks for evaluation and interpretation of generative models.

We used photometry and spectra from the SDSS.
While the SDSS provides a rich data set for deep learning,
it is limited to the optical and near-infrared wavelengths.
Future studies could explore combining multiple photometry sources by matching the SDSS objects to
photometric objects in other surveys, such as GALEX, WISE, or UKIDSS.
Future studies could also explore different strategies to improve the quality of generated images.
For example, although we used feature matching in this work to obtain a strong classifier,
if the goal is to improve the quality of generated images, an alternative technique
called minibatch discrimination will likely work better \citep{salimans2016improved,dai2017good}.
Finally, future studies could investigate the application of deep generative models in other settings,
such as unsupervised classification, object segmentation, and redshift estimation.

\clearpage
\bibliographystyle{plainnat}
\bibliography{thesisbib}

%\appendix*
%\include{Appendix.tex}

\backmatter

%\bibliographystyle{plainnat}
%\bibliography{thesisbib}

\end{document}
\endinput
%%
